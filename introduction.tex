\section{Introduction}

The presence of lower tropospheric aerosols has significant impacts on air quality and human health. Studies have shown that aggravated aerosol pollution in terms of fine particulate matter (PM2.5), contributes to over a million premature deaths annually \citep{lelieveld2015contribution,davidson2005airborne}. Additionally, aerosols play a crucial role in the energy balance of the Earth system, as they can scatter or absorb radiation and indirectly affect the formation, lifetime, and albedo of clouds. Anthropogenic activities, including human-made pollution, have contributed to a substantial increase in the tropospheric aerosol burden since the pre-industrial era, further intensifying these effects \citep{Bond07}. According to the report on the 6th Intergovernmental Panel on Climate Change (IPCC), the anthropogenic aerosol effective aerosol radiative forcing ranges from -0.63 to -1.37 W m-2 \citep{smith2020effective}. Furthermore, recent urban-scale studies have shown that anthropogenic aerosols may impact regional climate by affecting the urban heat island intensity \citep{han2020mechanisms,yang2020pm2,wu2017urban} and altering the precipitating systems over urban areas \citep{rosenfeld2008flood,van2007urban}. Despite the crucial role of anthropogenic aerosols in regional and global climate, significant uncertainties persist in their numerical simulation. This uncertainty is due to a limited understanding regarding their emissions, distribution in the atmosphere, and optical properties \citep{kinne2006aerocom,schulz2006radiative,textor2006analysis,myhre2013radiative,samset2013black}.

The accurate representation of anthropogenic aerosol emissions and their consistent time series data are crucial for Earth system models (ESMs) and atmospheric chemistry and transport models \citep{hoesly2018historical}. The prescribed emission time series serve as key inputs to these models, which use recent aerosol emissions as a starting point to predict future aerosol concentrations. However, most of these emission sources are spatially heterogeneous at regional and local scales, and their relatively short lifetime in the atmosphere results in a heterogeneous distribution, both geographically and vertically \citep{koch2009evaluation}. Retaining this heterogeneity during simulations is critical since model performance evaluation in areas such as aerosol formation, transport, cloud interactions, and deposition depends on the quality of the prescribed emissions used to drive the models. Using lower-resolution data may result in a loss of heterogeneity, leading to lower accuracy in the aerosol simulation and model evaluations, particularly near the sharp emission gradient zones. This error may be more significant for high-resolution model simulations, often used in urban-scale studies.

The U.S. Department of Energy (DOE) Energy Exascale Earth System Model (E3SM) is a state-of-the-art ESM that aims to produce actionable and accurate predictions of regional trends relevant to Earth system variability and change \citep{golaz2019doe}. The E3SM atmosphere model (EAM) has several parameterizations to represent different physical and chemical processes, including an aerosol emission treatment to handle prescribed emissions, which can impact the horizontal and vertical distribution of simulated aerosol concentration, their lifecycle, and interactions with clouds and radiation \citep{burrows2022oceanfilms,liu2012toward,liu2016description,wang2020aerosols}. However, the default emission treatment used in the standard and high-resolution EAM configurations has some limitations. Like many atmospheric models, EAM uses an unstructured cubed-sphere spectral-element (SE) grid due to its advantages over regular latitude-longitude (RLL) grids, such as offering high-resolution capabilities and improved computational scalability through Regionally Refined Mesh (RRM) \citep{taylor2010compatible,dennis2012cam}. Since conservative regridding within E3SM is not currently available, EAM linearly interpolates input emissions data to the model-native SE grid, which leads to a lack of mass conservation in current E3SM simulations. Furthermore, for low-resolution and RRM simulations, EAM uses low-resolution ($\sim$2$^{\circ}$) anthropogenic aerosol emission data, which fails to preserve the spatial heterogeneity in prescribed emissions. Therefore, improving the emission treatment in EAM is essential, especially for future E3SM applications with regional refinement at high resolutions, such as convection-permitting scales.

In this study, we introduce a new emission treatment for the Energy Exascale Earth System Model (E3SM) that ensures conservation of mass fluxes and preserves the original emission heterogeneity. Our objective is to assess the error estimates associated with the default emission treatment and the impact of resolved heterogeneity and mass conservation in high-resolution simulations. Our new treatment approach utilizes routines to assimilate emissions at the model-native Spectral Element (SE) grid, which makes it suitable for any resolution and applicable to both uniform resolution and Regionally Refined Mesh (RRM) configurations. This new treatment is crucial for implementing accurate emissions in future aerosol model evaluations at high-resolution urban to convection-permitting scale. 

This article is organized as follows: In Section 2, we provide a detailed account of the methods used, including the model configuration and experimental setups. In Section 3, we present the results of our study, including improved anthropogenic aerosol emission data, model comparison results, and an evaluation of our model against observations. Finally, we summarize our findings and provide conclusions in Section 4.
