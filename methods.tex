\section{Methods}
In this section, we provide description of the model used for this study, process of implementing the new emission treatment in the model, and the experimental design to evaluate the new treatment.

\subsection{Model Configuration}
In this study, we utilize the atmosphere component (EAMv2) of the Energy Exascale Earth System Model version 2 (E3SMv2) \citep{golaz2022doe}, which was developed by the United States Department of Energy (DoE), to investigate the impact of the new emission treatment on aerosol processes. E3SMv2 includes a comprehensive aerosol model, derived from the four-mode version of Modal Aerosol Module (MAM4), representing major anthropogenic and natural aerosol species, including black carbon (BC), primary organic matter (POM), secondary organic aerosol (SOA), marine organic aerosol (MOA), sulfate, mineral dust, and sea-salt in four lognormal size modes \citep{wang2020aerosols,liu2016description}. The model accounts for various aerosol processes including nucleation, coagulation, condensation, SOA formation, convective transport, wet removal, and dry deposition.

EAMv2 employs a spectral element dynamical core as in EAMv1 \citep{taylor2010compatible,dennis2012cam}, but utilizes “physics grid” (pg2 grid) for unresolved physics parameterization and a separate dynamics grid for resolved processes. This allows for an increased computational efficiency by reducing the effective resolution for the physics parameterization computations. The standard configuration of the model uses a low “ne30pg2” resolution (LR), which has a grid spacing of $\sim$110 km ($\sim$1$^{\circ}$) for dynamics and a grid spacing of $\sim$165 km ($\sim$1.5$^{\circ}$) for physics. E3SMv2 also supports fully coupled regionally refined mesh (RRM) configurations for high-resolution applications. One of the supported stable RRM configurations has a high-resolution mesh centered over North America (NA RRM). The NA RRM setup has a horizontal resolution of ne120pg2 ($\sim$28-km dynamics grid and $\sim$42-km physics grid) over North America and a horizontal resolution of ne30pg2 over rest of the globe (Fig. 1b). The high-resolution refined mesh is located approximately within 10$^{\circ}$N to 80$^{\circ}$N and 170$^{\circ}$W to 10$^{\circ}$W. This region covers a significant number of observation sites (AERONET and IMPROVE) for model evaluation (Fig. 1a). NA RRM has a total of 57816 computational elements or grid cells as opposed to 21600 in the standard configuration.

In the present study, we conduct experiments in both LR and NA RRM setup to explore the impact of the new emission treatment. More detailed description of the experimental setups is available in section 2.4.

\subsection{E3SM emission treatment}

The original source of anthropogenic emissions from agricultural, industrial, energy, transportation, and domestic sectors for EAMv2 are mostly derived from Community Emissions Data System (CEDS) inventory \citep{hoesly2018historical}. The biomass burning emission is derived from the fourth generation of the Global Fire Emissions Database (GFED4) \citep{giglio2013analysis,van2017historic}. CEDS provides historical (1750$-$2014) inventory of anthropogenic GHGs, reactive gases, and aerosols for Coupled Model Intercomparison Project phase 6 (CMIP6) \citep{eyring2016overview}. It includes anthropogenic emissions for the primary aerosol species, such as black carbon (BC) and organic carbon (OC), and chemically reactive gas sulfur-di-oxide (SO$_2$) as a precursor for Sulfate aerosols (SO$_4$). These data are in regular latitude longitude (RLL) grid at {0.5}$^{\circ}$\times{0.5}$^{\circ}$ resolution. GFED4 is also a part of the inputs for CMIP6, which provides historical (1750$-$2015) anthropogenic aerosol emission inventory of BC, OC, and SO$_2$ from biomass burning in RLL grid at {0.25}$^{\circ}$\times{0.25}$^{\circ}$ resolution. In the standard configuration of EAMv2, low resolution ({1.9}$^{\circ}$\times{2.5}$^{\circ}$) RLL gridded monthly historical (1850-2014) emission data are prescribed for ne30pg2 and RRM simulations. EAM also considers vertically distributed emissions from biomass burning, industrial, energy, and volcanic sources. These elevated emissions are distributed across 13 different altitudes, ranging from the ground level to approximately 7 km above the surface. The distribution of emission within each layer is uniform, but the distribution varies between layers, depending on the source of the emission. Elevated sulfur emissions from energy and industrial sectors are emitted at altitudes between 100 and 300 meters above the surface, while biomass burning emissions are distributed across all 13 altitude ranges, with diminishing fractions at higher elevations.

The emission treatment is a key component of the post-coupler processes. The EAMv2 utilizes consistent routines to process all prescribed anthropogenic aerosol and precursor gas emissions, where they are read from regular latitude-longitude gridded inputs, initialized either as emission fluxes or elevated sources, and are subsequently spatiotemporally interpolated to account for the model-native spectral element (SE) grid compatibility. The interpolated emission data are passed to the aerosol microphysics modules. The aerosol microphysics modules simulate the formation, growth, and removal of aerosols in the atmosphere, including processes of nucleation, coagulation, condensation, and sedimentation. During this stage, the elevated emissions are applied to the gas-phase chemistry calculations, and the surface fluxes are updated before dry deposition estimates are made (Fig. 2a).

EAMv2 considers emissions at the bottom layer of the model as surface emission fluxes and vertically distributed emissions as elevated sources. So, surface layer emissions and vertically injected emissions are handled separately (Fig. 2b). However, EAMv2 treats anthropogenic emissions and prescribed natural emissions (e.g., Dimethyl sulfate (DMS) ) in the same manner, and all computations are performed on the  model-native grid. The original EAMv2  can not handle emission data on  the model-native SE grid. Since all available emissions data sources are provided in RLL grid format, interpolation is needed to regrid the data into model-native grid structure. After the spatial interpolation, the model also performs temporal interpolation. In EAMv2, linear interpolation is applied to convert the prescribed RLL grid data to model-native grid. Although it’s convenient to linearly interpolate the same emission data to model grid at different spatial resolutions, the current treatment does not conserve mass. Also, when interpolated to higher resolutions, large emission errors will occur compared to the original emission data.

\subsection{New emission treatment}

To address the limitations of the default emission treatment in E3SM, we implemented a new emission treatment. The new treatment modifies the current treatment in the following ways:
\begin{enumerate}
\item	We have modified aerosol emission routines, which allows EAM to read prescribed anthropogenic aerosol emissions in both regular latitude-longitude (RLL) and model-native (SE) grids.
\item	To preserve spatial heterogeneity and mass conservation, we prepare the emission data on model-native grids offline using a conservative remapping tool using a high-resolution emission inventory as input. The online linear interpolation is switched off when the new treatment is used (reading the conservatively remapped data directly. 
\end{enumerate}
The new treatment has been implemented in both EAMv1 and EAMv2 for scientific evaluations. To simplify and automate the offline steps (Fig. 2b), we developed a Python wrapper package that utilizes TempestRemap \citep{ullrich2015arbitrary,ullrich2016arbitrary} and ncremap \citep{zender2008analysis}. This complementary grid generator tool (ggen) is used to generate grids, weights, and re-format the high-resolution emission data to the new emission treatment compatible format in model-native grid. This package is applicable to both surface and elevated emissions at any resolution.  

In this study, we use {0.63}$^{\circ}$\times{0.47}$^{\circ}$ RLL grid anthropogenic aerosol and precursor gas emission data as the high-resolution emission source, which were derived from the original CEDS and GFEDv4 data set and retains the injection heights for the vertically distributed emissions as in the default treatment. However, this new treatment can also be applied to other emissions data at any resolutions in RLL, model-native, and RRM grid formats. As will be shown later, the new treatment significantly improves the emission spatial heterogeneity in high-resolution applications.

\subsection{Experimental setup}

Two groups of E3SMv2 simulations at different horizontal resolutions were conducted to estimate the impacts of the new emission treatment in EAMv2 (Table 1).  Each group consists of control simulations using the default emission treatment and spectral-element (SE) treatment simulations utilizing the new emission treatment. The simulations were performed for both the low-resolution standard configuration of E3SMv2 (ne30pg2 or LR) and high-resolution (NA RRM) setup with 72 vertical layers (Fig. 1). For the control simulations, we follow the standard configuration and use the default low-resolution (1.9x2.5$^{\circ}$) prescribed emissions of aerosols and precursor gases in RLL grid. For SE simulations, we prepared anthropogenic aerosols and precursor gas emission on model-native grid from a high-resolution inventory. We conducted simulations with both Present-Day (PD, year 2014) anthropogenic aerosol emissions, and Pre-Industrial (PI, year 1850) anthropogenic aerosol emissions. All simulations include active atmosphere and land with prescribed monthly mean climatological SST and sea ice from 2005-2014.

All simulations were conducted using a meteorological nudging method \citep{sun2019impact,zhang2022further}, in which the horizontal wind components (u and v) were nudged to the European Centre for Medium-Range Weather Forecasts (ECMWF) ERA5 reanalysis \citep{hersbach2020era5}. These simulations were performed from October 1st, 2015, to December 31st, 2016. The first three months from the year 2015 were discarded as a model spin-up period, and the remaining 12 months were used for the analysis in this study. To perform these nudging simulations in EAMv2 at both LR and RRM resolutions, ERA5 reanalysis data from the year 2015-2016 were prepared for EAMv2 spectral-element (SE) dynamical core. These data in SE grid were used to nudge the simulation meteorology, using a relaxation timescale of 6 hours.

\subsection{Observation Data}

To evaluate the model's ability to simulate regional to local aerosols with the updated emission treatment, simulated values for the surface concentrations of Black Carbon (BC), Primary Organic Matter (POM), and Sulfate (SO$_4$) aerosols, and optical properties such as Aerosol Optical Depth (AOD) are compared to observational data from regional networks such as the Interagency Monitoring of PROtected Visual Environment (IMPROVE) and Aerosol Robotic NETwork (AERONET) (Fig. 1a). Only present-day (PD) simulations (LR-PD and RRM-PD) were used for the evaluations.

IMPROVE, is a network of aerosol monitoring stations located in protected areas in the United States, such as national parks and wilderness areas \citep{malm2004spatial}. This network measures aerosol properties such as surface concentrations, size distribution, composition, and optical properties. IMPROVE surface concentration measurements are only available over the United States and provide daily data 3 times a week. For this evaluation, we consider daily average surface concentration measurements from all available IMPROVE sites for each aerosol species in the year 2016. Measurements are available from over 150 sites for BC, organic carbon (OC), and sulfate aerosol surface concentration. We also multiply the observed OC by 1.4 before comparing against simulated POM. Since the IMPROVE measurements are for fine aerosol particles, we do not consider simulated aerosols in coarse mode. We also applied a conversion factor of 96/115 ($\sim$0.83) to the simulated sulfate concentration before comparing against the IMPROVE measurements.

AERONET, on the other hand, a global network of ground-based sunphotometers, measures aerosol properties such as size distribution, composition, and optical properties \citep{holben1998aeronet}. For this evaluation, we used AOD spectral radiometer daily mean measurements from over 120 active sites during the simulation year (e.g., 2016), which fall within the North America high-resolution mesh (bounded by 15$^{\circ}$N to 75$^{\circ}$N and 55$^{\circ}$W to 170$^{\circ}$W) in the RRM setup. 

The simulated daily mean data were spatiotemporally aligned with the observational data according to the time and location of each active observational site. These daily average data were conditionally sampled based on EAMv2 emission differences found between the default and new treatment (see details in section 3.3). We consider the same sites for RRM and LR simulation evaluations. Finally, monthly mean of these spatiotemporally collocated data were used to compare simulated and observed surface concentration and AOD.



