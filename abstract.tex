\begin{abstract}
Emissions of anthropogenic aerosol particles and their precursors are often prescribed in global aerosol models. Most of these emissions are spatially heterogeneous at model grid scales. When remapped from low-resolution data, the spatial heterogeneity in emissions can be lost, leading to large errors in the simulation. It can also cause the conservation problem if non-conservative remapping is used. The default emission treatment in Energy Exascale Earth System Model (E3SM) suffers from both problems. In this study, we introduce a new emission treatment for the E3SM atmosphere model (EAM) that ensures conservation of mass fluxes and preserves the original emission heterogeneity. We assess the error estimates associated with the default emission treatment and the impact of resolved heterogeneity and mass conservation in both globally uniform standard-resolution ($\sim$100 km) and regionally refined high-resolution (< 50 km) simulations. The default treatment incurs significant errors near surface, particularly over sharp emission gradient zones, with larger errors in high-resolution simulations. The default treatment significantly underestimates (overestimates) aerosol burden, surface concentration, aerosol sources (aerosol sinks) over highly polluted regions, while overestimating (underestimating) them over nearby less polluted regions. Large errors can persist at higher elevation from daily mean estimates, which can affect aerosol extinction profiles and aerosol optical depth (AOD). Our new treatment significantly improves the accuracy of the aerosol emissions from surface and elevated sources near sharp spatial gradient regions, with significant improvement in the spatial heterogeneity and variability of simulated surface concentration in high-resolution simulations. We utilized routines to assimilate emissions at the model-native Spectral Element (SE) grid, making it suitable for any resolution and applicable to both uniform resolution and Regionally Refined Model (RRM) configurations. Our findings suggest the new emission treatment is crucial for future E3SM applications with regional refinement at high resolutions, such as convection-permitting scales, to help improve the simulated aerosol lifecycle and the aerosol impact on climate.
\end{abstract}