\section{Conclusions}
In this study, we present a new and improved emission treatment designed to enhance the accuracy of the anthropogenic aerosol emissions data used in E3SM LR and RRM simulations. However, the treatment is applicable to other models, emission sources, and resolutions. The default EAM emissions treatment fails to preserve spatial heterogeneity or conserve mass, resulting in substantial errors. The improved treatment accurately preserves the original spatial heterogeneity both inland and near the coast, providing a more accurate representation of surface and vertically distributed emissions. Our results indicate that the default treatment for anthropogenic aerosol emissions in both LR and RRM simulations consistently yields large root mean square errors (RMSE) for all species. The improved emissions data prepared for the new emission treatment can resolve these issues by maintaining spatial heterogeneity and mass conservation.

Significant differences exist between the new and default emission treatments in terms of aerosol surface concentration and burden. The differences are closely linked to the spatial distribution of prescribed emission gradients and are found to persist at higher elevations above the surface, with larger discrepancies observed in high-frequency concentration profiles. The findings suggest that errors in default emissions can have a significant impact on the accuracy of high-resolution urban-scale simulations. The default treatment can also yield larger errors in simulated extinction and absorption profiles near and above the surface, leading to significant errors in high-frequency aerosol optical depths. This result is particularly relevant, given that short-term high-frequency fields are often used for urban-scale studies and model evaluations against observations.

Using the default emission treatment can lead to significant errors in aerosol sources and sinks. Substantial errors in the simulated aerosol sources, sinks, and their decomposed components are found, which were consistent with the spatial distribution of the relative differences. The prescribed emissions and dry deposition are the main causes of larger errors in BC and POM source-sink components. Interestingly, errors for wet deposition components for POM and BC were significantly smaller than dry deposition, despite the large fractional contribution to sink. This is due to the fact that in-cloud scavenging routines in EAMv2 do not directly affect prescribed BC and POM in primary carbon mode. The largest error contributions for sulfate aerosol sources come from gas-aerosol exchange and in-cloud aqueous-phase chemistry, while wet deposition, particularly stratiform in-cloud scavenging, is the biggest contributor to errors in sinks. 

The new emission treatment leads to improved heterogeneity in simulated surface concentration, particularly in regions with sharp emission gradients. Furthermore, simulations with the new treatment consistently outperformed simulations with the default treatment in terms of seasonal mean biases and variability of daily mean surface concentrations of black carbon and primary organic matter. Our study highlights the importance of maintaining spatial heterogeneity in high-resolution simulations, and therefore recommends that atmospheric models adopt the new emission treatment for improved accuracy in geographical and vertical distribution of aerosol concentrations, their sources, and sinks.
