\section{Results and Discussions}

\subsection{Improving emissions in EAM}

One of the primary goals of our new emission treatment is to enhance the accuracy of the emissions data utilized in the standard LR and RRM simulations. The default EAM emissions treatment fails to preserve spatial heterogeneity or conserve mass, resulting in substantial errors (as described in section 2.2). Figure 3 illustrates the spatial distribution of total emitted surface Black Carbon (BC) and SO$_2$ over the Eastern United States in 2014. Column integrated SO$_2$, the primary precursor to Sulfate aerosols, depicts the spatial distribution of elevated sources in EAMv2. Figure 3 indicates how well the heterogeneity of prescribed surface and elevated emissions are represented in the standard RRM simulations when compared against the original high-resolution data.

The original high-resolution data display highly heterogeneous emissions over the land surface, largely driven by industrial, energy, and transportation sectors (Figure 3a, d). As expected, the default emission treatment fails to capture much of the heterogeneity over sharp emission gradient zones (Figure 3b, e). For example, panel b and e depict seven major cities (e.g., Boston, New York, Chicago, Toronto, Montreal, Los Angeles, and San Francisco) with large anthropogenic BC and SO$_2$ emissions respectively. The default treatment not only significantly underestimates emissions ($\sim$80\%) from those cities, but also grossly misrepresents emissions in the nearby regions. This can severely impact the accuracy of high-resolution studies over urban regions. Additionally, the default treatment provides an inaccurate representation of emissions near the coasts, where regions of BC sources from shipping sectors are misrepresented by comparatively large emissions from transport and industrial sectors. In general, both surface and vertically distributed emissions in the default treatment fail to maintain emission land-sea contrast near the coastlines. In contrast, panels c and f illustrate the spatial emissions distributions in the improved treatment, which accurately preserves the original spatial heterogeneity both inland and near the coast. Supplementary Fig. S1 depicts the large error regions in terms of the emission difference between the improved and default treatment from surface and elevated sources. It also indicates that larger errors, from sharper spatial gradients, exist near regions with larger emission. For instance, the difference from surface BC and POM is much larger over the Eastern US, which is consistent with the larger surface emissions over the Eastern US. This is important to separate the regions with sharper spatial gradients in the later sections. These improved datasets, which retain both heterogeneity and mass conservation, provide a more accurate representation of surface and vertically distributed emissions in EAMv2. It also accurately applies prescribed emissions in major cities, making the new treatment suitable for urban-scale studies.

To evaluate the loss of accuracy of the EAMv2 emissions in the default treatment, we calculate error estimates against more accurate emissions data, which preserve the original spatial heterogeneity and mass conservation. Table 2 provides a summary of the error estimates for Black Carbon (BC), Primary Organic Matter (POM), and Sulfate (SO$_4$) aerosol emissions data used in the EAMv2 RRM simulations, including area-weighted spatial mean (Mean), normalized mean bias (NMB), standard deviation (StdDev), normalized standard deviation (NStdDevB), root mean square error (RMSE), and normalized RMSE (N\_RMSE) for the present-day (year 2014). The values for elevated sources are estimated over column integrated spatial distributions. Table S1 presents the error statistics for emissions in the LR simulations. The RRM configuration in this study has a high-resolution mesh over North America (NA), with the largest sources of emissions occurring over the land and only emissions from the shipping sector occurring over the ocean. Therefore, we consider NA land surfaces bounded by 15N to 75N and 55W to 170W for the error estimates.

Our results indicate that the default treatment for anthropogenic aerosol emissions in both LR and RRM simulations consistently yields large root mean square errors (RMSE) for all species. It is worth noting that the normalized mean bias (NMB), which indicates mass conservation errors, is generally small (< 1\%) for long-term global estimates. However, it can be considerably larger for regional estimates, such as BC over the northeast United States, where it can reach up to 25\% (not shown). For North American land, it varies from $\sim$1 to $\sim$10\% for RRM and $\sim$0.3 to $\sim$2.5\% for LR emissions. Since NMB is influenced by the magnitude of emissions, it tends to be larger for anthropogenic emission sources than for biomass burning sources. As a result, we observe larger NMB for surface emissions of BC and POM. On the other hand, it is larger for elevated sources of SO$_4$ emissions, which includes vertically distributed anthropogenic sources from industry and energy sectors as well as biomass burning sources. This is consistent with larger sulfate emission differences from elevated sources between the improved and default treatment (Fig. S1).

Regardless of the NMB values, we find that both surface and elevated sources exhibit large RMSE values. To compare the RMSE values across different species and sources, we use normalized RMSE (N\_RMSE) values. We found consistently larger N\_RMSE values (ranging from 54\% to 84\%) for EAMv2 RRM emissions compared to the LR emissions (ranging from 34\% to 57\%), with the largest N\_RMSE for Sulfate aerosol elevated sources.

Overall, our findings suggest that the default treatment for anthropogenic aerosol emissions in LR and RRM simulations leads to large errors. Improved emissions data prepared for our new emission treatment can resolve these issues by maintaining the spatial heterogeneity and mass conservation.

\subsection{Model-to-model Comparison}

In this section, we compare the simulated fields between SE-PD and PD simulations to evaluate the error estimates from the default emission treatment and the impact of implementing the new emission treatment. We constrain these comparisons within regional to local scales over North America to encapsulate large differences within the high-resolution RRM mesh.

\subsubsection{Simulated aerosol concentration}
In Fig. 4, we present the spatial distribution of surface concentration resulting from RRM-PD simulation using the default emission treatment, along with the relative differences between RRM-SE-PD and RRM-PD. The results show significant differences over North America with a normalized root mean squared error (N\_RMSE) of 39\%, 34\%, and 12\% for BC, POM, and sulfate aerosols, respectively. While the absolute relative differences of BC and POM surface concentrations can reach up to 50\% in some regions, sulfate aerosols exhibit weaker relative differences ranging from 2-10\%. This can be attributed to the fact that prescribed sulfate emissions are mostly emitted from elevated sources, such as industrial and energy sectors, as opposed to BC and POM emissions, which are primarily emitted at surface level. Furthermore, significant differences were found in simulated surface concentrations for LR simulations, with normalized RMSE of 19\%, 15\%, and 8\% for BC, POM, and sulfate aerosols, respectively (Fig. S2). The weaker N\_RMSE in LR simulations compared to RRM simulations is consistent with the weaker N\_RMSE found from the default surface emissions used in the LR experiments (Table S1).

Simulated surface concentration differences exhibit positive and negative bias regions, indicating patterns of sharp spatial gradients. Although these patterns appear randomly distributed across North America, they are closely linked to the differences between prescribed emissions from the new and default treatment (Fig. S1). To confirm this, we separated North America into three distinct regions based on the prescribed emission differences between the RRM-SE-PD and RRM-PD simulations. Masking was applied to distinguish regions with strong and weak errors from default emissions. Regions with aerosol emission differences above the 75th percentile, below the 25th percentile, and within the 25th/75th percentiles over North America were selected.

Figure 5 shows relative differences of different simulated fields, including aerosol burden, surface concentration, net aerosol sources, and sinks. Each field is masked based on their respective aerosol emission difference beyond and within the 25th/75th percentiles. Sharp spatial gradients are predominantly found near highly polluted regions, and fields masked by emission differences above (below) 75th (25th) percentiles reveal larger positive (negative) relative differences. While the estimates vary among different aerosol species and fields, the relative difference can range from -90\% to over 50\%. In contrast, simulated fields masked by emissions within 25th/75th percentiles, representing weaker emission gradients, show significantly smaller relative differences, ranging from 0.3 to $\sim$5\%. These results suggest that the seemingly random distribution of relative differences in Fig. 4 is strongly linked to the emission differences between the new and default treatment. It also indicates that errors in default emissions not only impact simulated surface concentrations but also aerosol burden (weak) and their source-sinks (strong). A decomposed source-sink analysis is described in a later section.

Figure 5 also displays relative differences from major cities over North America with large anthropogenic aerosol emissions (as depicted in Fig. 3). These cities are located above the 75th percentile masked regions and display similar patterns, with significant positive or negative biases in the simulated aerosol burden, surface concentration, and aerosol source and sinks. This finding suggests that errors arising from default emissions can have a significant impact on the accuracy of high-resolution urban-scale simulations. We note that the simulated aerosol burden yields weaker relative differences compared to surface concentration. This is expected since we are analyzing long-term annual means of column integrated concentration (burden), which are influenced by several other processes, such as condensation-aging, coagulation, aqueous-phase cloud chemistry, depositions, vertical diffusion, and horizontal transport. 

While the column integrated concentration (i.e., aerosol burden) may be less affected, larger discrepancies can arise at the surface level and at higher elevations in high-frequency concentration profiles. Figure 6 exhibits significant differences in simulated daily mean BC concentration profiles and column integrated burden. Panels a-d (e-h) illustrate vertically distributed aerosol concentrations and column-integrated burden from highly (nearby less) polluted locations to demonstrate the simulated biases near sharp spatial emission gradient zones. The vertical profiles indicate that the larger differences can persist at higher elevation above the surface. Over polluted regions, the default emission treatment can significantly underestimate surface and elevated aerosol concentration within the boundary layer. On the other hand, over comparatively cleaner nearby regions, the default emission treatment can significantly overestimate surface and elevated aerosol concentrations within the boundary layer. Due to the high variance in daily mean fields, as opposed to long-term monthly or annual means, we found significantly larger errors in simulated high-frequency data. For instance, relative differences in simulated BC burden could reach up to 25-30\% on certain days (Fig. 6d).

\subsubsection{Aerosol optical properties}
Figure 7 illustrates the spatial distribution of simulated aerosol extinction and absorption near the surface, as well as the Aerosol Optical Depth (AOD) and absorption AOD (AAOD). From long-term means, larger differences in simulated aerosol concentration are observed near the surface. Therefore, larger relative differences in annual mean vertical profiles of extinction and absorption are constrained near the surface level with a normalized RMSE of 13\% and 29\% respectively. Our new treatment improves the anthropogenic aerosol emissions, leading to larger differences in simulated absorption profiles near the surface compared to extinction profiles. Extinction profiles are influenced by natural aerosols such as sea-salt and dust, in addition to the anthropogenic aerosols.  Notably, aerosol absorption profiles near the surface can reach a relative difference of approximately 50\% over major cities, southern Mexico, and northwest North America, which is consistent with the spatial distribution of emission differences between the new and default treatment.

AOD and absorption AOD are column-integrated aerosol extinction and absorption, respectively, that are strongly influenced by processes such as aerosol chemistry, horizontal transport, and vertical diffusion. Our results indicate that the annual mean spatial distributions of these simulated fields do not show significant differences, with some exceptions over northwest North America, which can be attributed to the unusually high biomass burning emissions of BC and POM during the 2014 Northwest Territories (NWT) fires. The summer of 2014 was the most severe fire season in NWT history, resulting in wildfires burning a record 3.4 Mha and an estimated emission of 164 ± 32 Tg of carbon into the atmosphere \citep{veraverbeke2017lightning,kochtubajda2019assessment}. Biomass burning emissions are prescribed as elevated sources of anthorpogenic aerosols in E3SM. Substantial errors from default emissions persist at higher elevation (Table 2). Since the present-day simulations are conducted using the emissions from the year 2014, simulated AAOD show large relative difference over NWT, which may be driven by the persisting errors in elevated BC and POM emissions from default treatment. 

For high-frequency daily mean fields, AOD and absorption AOD can have normalized RMSE values of approximately 15-20\% over North America. Figure 8 shows the time series of high-frequency daily mean extinction and absorption profiles, as well as the time evolution of AOD and absorption AOD (AAOD) during the month of July 2016. Consistent with the persistent aerosol concentration differences found at higher elevations, significant differences exist in the simulated profiles within the boundary layer. Although the long-term annual mean AOD and AAOD do not display significant errors over the highly polluted northeast US region, relative differences from daily mean estimates could reach up to approximately 10-12\% on certain days. Our results highlight the potential impact of errors from default emission treatment on high-frequency aerosol concentrations at higher elevations, which can further influence aerosol extinction profiles and lead to significant errors in simulated aerosol optical depth. This finding is particularly relevant as short-term high-frequency fields are often used for urban-scale studies and model evaluations against observations.

\subsubsection{Decomposed source-sink analysis}
Our results in section 3.2.1 suggest that using the default emission treatment can lead to significant errors in aerosol source and sinks, despite having a weaker impact on column integrated aerosol burden (Fig. 5). In this section, we conduct an aerosol source-sink analysis to estimate the potential errors propagated from default emissions into different processes during the RRM simulations. Figure 6 presents stacked bar plots indicating the fractional distribution of major processes contributing to the total simulated aerosol sources and sinks in the EAMv2 present-day RRM simulation. The analysis considers the North American region (15$^{\circ}$N to 75$^{\circ}$N and 55$^{\circ}$W to 170$^{\circ}$W) and estimates the percent contributions from annual means of each component that drives the simulated sources and sinks. The overall fractional distribution is same for LR simulations (not shown). 

The stacked bar plots in Fig. 6a reveal that prescribed emissions are the primary drivers of BC and POM sources, with a surface (from anthropogenic sources) to elevated (from biomass burning sources) emission ratio of 70:30 and 24:76, respectively. In contrast, only about 5\% of the total sulfate sources are driven by the prescribed sulfate emissions. Gas-aerosol exchange and in-cloud aqueous-phase (SO$_4$) chemistry contribute to sulfate sources by $\sim$22\% and $\sim$71\%, respectively. BC and POM sinks are evenly modulated by dry and wet depositions, with turbulent dry deposition accounting for most of the dry deposition ($\sim$41\% and $\sim$36\%), and in-cloud scavenging accounting for most of the wet deposition ($\sim$56\% and $\sim$60\%). Stratiform clouds modulates the larger portion of the in-cloud wet deposition. Conversely, sulfate removal has an uneven $\sim$15\% and $\sim$85\% contributions from dry and wet depositions respectively, with largest contribution from stratiform in-cloud scavenging ($\sim$65\%). 

Figures 10 and 11 present the error statistics for simulated source-sink between RRM-SE-PD and RRM-PD in terms of weighted normalized RMSE. The weights are determined by the fractional contributions of each process shown in Fig. 9, and the actual RMSE and N\_RMSE for each process can be found in Table S2. The results indicate substantial errors in the simulated aerosol sources, sinks, and their components, which are consistent with the spatial distribution of relative differences shown in Supplementary Fig. S3-S9.

The normalized RMSE for both BC and POM total sources is approximately 71\%. However, the error contributions from anthropogenic sources are much larger for BC, with a contribution of about 49\%, compared to POM, which has a contribution of about 16\%. POM errors are primarily driven by the biomass burning sources. This difference is due to the spatial variability and magnitudes of biomass burning emissions (BB) in 2014, which are reflected in the surface and elevated emission ratios of BC and POM. It should be noted that the contribution from BB sources (e.g., elevated emission) is comparatively significant over NA due to the 2014 Northwest Territories fires. For BC, larger BB sources are concentrated within Northwest NA, while anthropogenic sources are prevalent over the rest of NA (Supplementary Fig. S3). In contrast, for POM, the BB sources has significantly larger contribution and the large anthropogenic sources are concentrated over the eastern and southern NA (Fig. S5). These source contributions are reflected into the spatial distribution and magnitude of relative differences, resulting in an uneven error contribution to BC and POM from different emission sources.

Total simulated aerosol sink yields a normalized RMSE of $\sim$45\% and $\sim$34\% from BC and POM respectively, with largest contributions from dry deposition (Fig. 10). This is consistent with the spatial distribution of relative differences over NA between RRM-SE-PD and RRM-PD in Fig. S4 and S6. Fundamentally, dry deposition rate ($F_{d}$) at height z can be linked to the vertical deposition velocity ($V_{d}$) and aerosol concentration (C) at that height as follows:

\begin{equation}
F_{d} = C(z) \times V_{d}
\end{equation}

$V_{d}$ in EAMv2 depends on gravitational settling and turbulent deposition velocities, where turbulent deposition velocities are calculated using surface layer information of the model \citep{zhang2001size}. Gravitational settling velocity ($V_{g}$) is derived from stokes’ settling velocity. It is calculated at all model levels and is defined as:

\begin{equation}
V_{g} = \frac{d_{p}^{2}g\rho_{p}C_{C}}{18\mu_{a}}
\end{equation}

Where g is the gravitational acceleration, $d_{p}$ is the particle diameter, $\rho_{p}$ is the particle density, $\mu_{a}$ is the air dynamic viscosity, and $C_{C}$ is the Cunningham slip correction factor \citep{seinfeld1998air}. Turbulent deposition velocity ($V_{T}$) considers $V_{g}$ at surface level in addition to the deposition through Brownian diffusion, impaction, and interception. It is defined as:
 
\begin{equation}
V_{T} = \overline{V_{g}} + \frac{1}{R_{a}+R_{s}+V_{g}R_{s}R_{a}}
\end{equation}

Where, $\overline{V_{g}}$ is the gravitational settling velocity at surface level, $R_{a}$ is the aerodynamic resistance and $R_{s}$ is surface resistance. $R_{s}$ is inversely proportional to the summation of collection efficiencies from Brownian diffusion, impaction, and interception. Gravitational and turbulent deposition fluxes at bottom are apportioned as follows:

\begin{align}
    F_{T} = F_{d} \times \frac{V_{T}}{V_{T}+V_{g}} \\
    F_{g} = F_{d} \times \frac{V_{g}}{V_{T}+V_{g}}
\end{align}

Since $V_{T}$ is substantially larger than $V_{g}$ at surface layer, most of the error contributions originate from the turbulent dry deposition fluxes. Our results show that over 99\% of the normalized RMSE in dry deposition comes from turbulent dry deposition flux, which is consistent with the fractional distribution of sinks in Fig. 9.

Normalized RMSE for wet deposition components for POM and BC are significantly smaller than dry deposition, which is inconsistent to the fractional contribution of sinks. EAMv2 aerosol wet deposition routines consider both in-cloud and below-cloud scavenging by stratiform and convective precipitation (Barth et al., 2000; Rasch et al., 2000). In-cloud scavenging is the dominant driver of total wet deposition. To investigate the inconsistencies, we focus on the stratiform and convective in-cloud scavenging routines in EAMv2. In-cloud scavenging through stratiform (convective) clouds consider only cloud-borne (cloud-borne + interstitial) aerosol particles. For convective in-cloud scavenging, the wet deposition routine incorporates a tuning factor and a convective cloud activation fraction, which ranges from 0.0 for primary carbon mode to 0.8 for other aerosol modes. Lower value indicates lower hygroscopicity. Since prescribed emissions of BC and POM are in primary carbon mode, they are weakly impacted by convective in-cloud scavenging. However, accumulation mode BC and POM formed through condensation-aging and coagulation are more strongly affected by the convective in-cloud deposition routine. Stratiform in-cloud routine strongly affects stratiform cloud-borne BC and POM, formed via droplet nucleation, in accumulation and coarse mode. Since neither of the in-cloud scavenging routines directly affects the prescribed BC and POM in primary carbon mode, we observe a weaker error contribution from wet deposition, despite its strong contribution to the total aerosol sinks.

For simulated sulfate aerosols, source and sink yields a normalized RMSE of $\sim$36\% and $\sim$9\% respectively (Fig. 11). The error estimates for sources align with the fractional distribution of each component, with largest contributions from gas-aerosol exchange and in-cloud aqueous-phase (SO$_4$) chemistry. Sulfate sinks yield a significantly smaller errors compared to BC and POM, consistent with the weaker contribution from dry deposition. Interestingly, we see larger error contributions from wet deposition, with stratiform in-cloud scavenging accounting for more than half of it. Prescribed sulfate aerosol emissions are in the Aitken and accumulation mode, which can be directly impacted by convective in-cloud scavenging. On the other hand, sulfate production through in-cloud aqueous-phase chemistry is attributed to cloud-borne aerosol particles in stratiform clouds. These particles are subsequently removed through stratiform in-cloud scavenging, which is the largest contributor to total wet deposition. The error estimates for each sink component are consistent with their fractional contributions to the total simulated sulfate sink. Therefore, we see larger normalized RMSE from in-cloud wet depositions for sulfate aerosols.

Our analysis highlights significant errors in simulated aerosol sources and sinks when the default emission treatment is used. Larger simulation errors are evident for BC and POM source-sink components, primarily due to the prescribed emissions and dry depositions.

\subsubsection{Anthropogenic aerosol forcing}
To evaluate the impact on anthropogenic aerosol radiative forcing, we investigate the decomposed aerosol effective radiative forcing ($\Delta{F}$) using the methods proposed by Ghan (2013) from the present-day (PD) and pre-industrial (PI) simulations. Figure S10 indicates significant errors in the spatial distribution of $\Delta{F}$ exist when using the default emission treatment. Over North America, the net aerosol forcing ($\Delta{F}$) is primarily driven by the indirect aerosol effects or aerosol-cloud interaction (ACI) terms, particularly the indirect shortwave forcing ($\Delta{F_{SW}}$). Using the default treatment can yield a normalized RMSE of 49\%, 51\%, and 60\% over NA in estimated net aerosol forcing, indirect aerosol forcing, and indirect shortwave forcing respectively. We note that these error estimates are representation of errors in spatial distribution of the aerosol forcing terms. The large differences found are mostly away from aerosol emission sources. In contrast, the relative difference in annual mean aerosol radiative forcing between simulations with the new and default emission treatment over the North America is only about 3-5\%. This is expected and consistent with previous long-term estimates. 

\subsection{Model evaluation against observations}
To demonstrate the improved heterogeneity in the SE-PD simulations resulting from the new emission treatment, we ideally require observational sites located near sharp emission gradient zones, which capture the larger errors in the default emissions. Figure S1a displays the spatial distribution of the absolute difference in BC emissions. where the sharpest gradients are found over the northeast US, with reduced errors over central to western US, where the emission spatial gradient is weaker. As a result, many observational network sites do not fall within or near the regions with large errors.

To address this issue, we have applied a conditional sampling approach based on the surface and elevated emission differences between the new and default treatment. Specifically, we consider sites that fall in the locations where the emission differences are above (below) the 75th (25th) percentiles. We found this approach yields similar results using the 90th/10th percentiles, but with a reduced number of sites. Without conditional sampling, sites falling within the 75th/25th percentiles can mask the impact of the new treatment (not shown). Conditional sampling raises the likelihood of selecting sites over large error regions, assuming larger gradients occur near larger biases.
 
The LR-PD and LR-SE-PD simulations did not show any significant differences in surface concentration or AOD (as shown in Fig. S11). Emissions at low resolution (with a grid spacing of $\sim$165 km for physics) used in LR simulations, cannot fully represent the highly heterogeneous regions of the original emissions. As a result, default LR emissions show significantly lower errors compared to default RRM emissions in terms of normalized RMSE (Table 1 vs Table S1). Low resolution in addition to the lack of appropriate observational sites, the scatter plots for LR-SE-PD simulation did not show any significant improvements.
 
In Fig. 12, we compare simulated aerosol surface concentration and Aerosol Optical Depth (AOD) from RRM-PD and RRM-SE-PD experiments to observational measurements from IMPROVE and AERONET sites (described in section 2.5). Similar comparison for LR simulations is shown in Fig. S2. Figure 12 shows substantial difference in simulated surface concentration of Black Carbon (BC) and Primary Organic Matter (POM) between RRM-PD and RRM-SE-PD. The spatial correlations between the simulated and observed BC (POM) surface concentration from RRM-PD and RRM-SE-PD are 0.44 (0.43) and 0.59 (0.51) respectively. Our results suggest significant improvements in spatial heterogeneity in terms of spatial correlation coefficient for BC (with Fisher’s Z of -1.8 at p ≤ 0.05) and potential improvements for POM (with Fisher’s Z of -1.2 at p ≤ 0.1).

In Fig. 13, we present the distribution of daily mean surface concentrations of BC and POM using the conditionally sampled sites (as in Fig. 12). It shows that RRM-PD consistently underestimates the seasonal means and the variability of observed daily mean measurements. This underestimated variance can be attributed to inaccurate representation of spatial heterogeneity in the default emission treatment. RRM-SE-PD improves the seasonal mean biases along with variability in simulated daily mean surface concentration. We acknowledge that using emissions from the year 2016 (which were unavailable for EAMv2) for the present-day (PD) simulations could yield more accurate bias comparisons (as discussed in Section 2.3). However, the changes in total emissions are much smaller (< 10\%) and are unlikely to significantly impact the simulated surface concentrations. For example, the total anthropogenic BC emissions over North America decreased from 0.3 Tg/yr in 2014 to 0.28 Tg/yr in 2016 (around a 5\% change). Therefore, the small year-to-year variation from 2014 to 2016 will not have a strong impact on the bias estimates.
 

EAMv2 overestimates sulfate (SO$_4$) aerosol surface concentration for all simulations with a normalized bias exceeding 130\% (Fig. 14). We note that the bias depends on the observation period used for the evaluation, and a long-term mean yields a lower bias. The spatial correlations (e.g., $\sim$0.72) between simulated and observed sulfate surface concentration from RRM-SE-PD show no significant improvement over RRM-PD. This lack of improvement is likely due to the distribution of sulfate and their precursor (SO$_2$) emissions in EAMv2. About 80\% of the total sulfate and SO$_2$ emissions prescribed in EAMv2 are from vertically distributed sources, mainly industrial and energy sources ($\sim$95\%) and are prescribed at 100-300 meters above the surface. Large differences exist for elevated sources as opposed to the surface sources (Fig. S1 c, d). Less than 20\% of the total sulfate and SO$_2$ emissions over North America, primarily from transportation, and residential sectors, which are emitted to the surface layer that will immediately affect the surface concentration. This leads to a weaker sensitivity to emission changes in simulating sulfate surface concentration. As expected, aerosol optical depth (AOD) scatter plots do not show any improvements either, with a spatial correlation of $\sim$0.35 for both RRM-PD and RRM-SE-PD. Since AOD is the column-integrated aerosol extinction, it is strongly modulated by other processes, such as chemistry, horizontal transport, and vertical diffusion. However, we found significant differences may exist in near surface extinction profiles between RRM-PD and RRM-SE-PD simulations (as shown in section 3.2.2).

Since large errors from default treatment are mostly outside the locations of available observational sites, we rely on conditional sampling for the evaluation. Our results suggest that the new emission treatment can significantly improve the spatial heterogeneity and daily variability in magnitudes of simulated surface concentration of aerosol species that are primarily emitted from surface in high-resolution simulations.


